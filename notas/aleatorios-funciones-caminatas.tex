\chapter{Números aleatorios, funciones, y caminatas aleatorias}

La meta de este capítulo es la de llegar a hacer una \defn{simulación} de uno de
los sistemas físicos más sencillos: una \defn{caminata aleatoria}.

Por una simulación, nos referimos a un modelo computacional que reproduce las
características principales de un sistema físico adentro de la computadora.

\section{Caminatas aleatorias}


Empecemos con el estudio de las caminatas aleatorias.
En este tipo de modelo, una partícula brinca en distintas direcciones al azar.
% Podemos pensar en un borracho que no puede controlar sus movimientos. Pero
Este modelo es una aproximación, más o menos buena, del comportamiento de muchos
sistemas físicos, por ejemplo el movimiento de una partícula grande
``Browniana'' en agua. 
El comportamiento ``aleatorio'' provee de los miles de colisiones que sufre la
partícula con las moléculas a su alrededor, cuyo efecto total modelamos como 
impulsos aleatorios.
Además, es una primera aproximación a una descripción del
tipo de movimiento que ejecuta un organismo vivo, por ejemplo una célula, una
hormiga, o un animal, viajando en su entorno.

Para simplicidad, consideremos inicialmente el caso donde el caminante brinca
en momentos de tiempo \defn{discretos} $n=1, 2, \ldots$, correspondiendo a
momentos de tiempo $\delta t$, $2 \delta t$, etc., donde $\delta t$ es un
incremento de tiempo pequeño.
% 
% los modelos que
% estudiamos serán \emph{discretizados} sobre una red.
% Por el momento usaremos una red regular en una o dos dimensiones; después
% extenderemos nuestro estudio a redes más complejas.

\section{Caminata aleatoria discreta en 1 dimensión}
Empecemos con una caminata aleatoria  que vive en una 
una \defn{red} en 1 dimensión, y una partícula --un
caminante-- que empieza en el orígen $0$ al tiempo $n=0$.
En cada paso el caminante ejecuta una de tres acciones con distintas
\emph{probabilidades}
\begin{itemize}
 \item  brinca a la derecha con probabilidad $p$;
\item brinca a la izquierda con probabilidad $q$; ó
\item se queda en el mismo lugar con probabilidad $r := 1-(p+q)$.
\end{itemize}
En el caso más sencillo, tiene que brincar, con $r=0$. El caso más clásico es
con probabilidades iguales de brincar en cualquier de las dos direcciones, es
decir, $p=q=\textstyle \frac{1}{2}$.

Este problema es uno de los problemas más clásicos en la física estadística, y
muchos de sus propiedades se conocen de manera analítica.

\section{Números aleatorios}

Para abarcar la \defn{simulación} de una caminata aleatoria, necesitamos poder
hacer elecciones entre distintas opciones con cierta \defn{probabilidad}.
Para hacerlo, utilizaremos \defn{números aleatorios}.

>Cómo se pueden generar números aleatorios en una computadora? La respuesta es
que <no se puede!, ya que una computadora puede llevar a cabo solamente
operaciones deterministas. Sin embargo, podemos generar secuencias de números
que se comportan ``como si fueran'' aleatorios; éstos se llaman números
\defn{pseudo-aleatorios}. Para hacerlo, ocuparemos secuencias generadas por
iteraciones deterministas, con ciertas propiedades escogidas para dar la
apariencia de aleatoriedad.

\section{Generación de número aleatorios}

Para el uso en una simulación seria, se emplearía una librería, por ejemplo el
Gnu Scientific Library. (En
particular, hoy en día se usa mucho el algoritmo llamado Mersenne Twister.)
Sin embargo, para un uso básico, es útil saber cómo se pueden generar los
números
aleatorios en una computadora.

La manera más frecuente de generar números pseudo-aleatorios es a partir de una
iteración $x_{n+1} = f(x_n)$, donde $x_n$ es \defn{entero}. Dependiendo de la
función $f$ se pueden obtener resultados que parezcan aleatorios.

Una de las secuencias que se solía usar era $f(x_n) = ax_n + b\ (\mathrm{mod\ }
m)$. Si se escogen bien la $a$, $b$ y $m$,  entonces esta iteración sencilla
puede dar buenos resultados. Este método recae en el hecho de que los enteros
en C++ son de un tamaño de 32 bits, entonces cualquier operación que da un
resultado mayor que este tamaño se truncará.

Un ejemplo está dado por
\begin{equation}
 a = 1664525; \quad b=1013904223; \quad m=2^{32}.
\end{equation}
El valor de $m$ es uno más que el máximo valor que se puede guardar en un
entero de 32 bits, por lo cual se puede ejecutar de manera automática el
módulo, al declarar las variables como 
\begin{lstlisting}
unsigned int x;
\end{lstlisting}
es decir, enteros que no pueden ser negativos.

\ejercicio
Implementa este método para generar números aleatorios enteros.
>Cómo se puede convertir en un método para generar números aleatorios en el
intervalo $[0,1]$? >En el intervalo $[0,1)$?

\ejercicio
Dibujar pares de estos números aleatorios en $[0,1]$ en un plano para verificar
que no hay ninguna correlación que se pueda detectar a ojo.






 Hoy en día se usa el método de ``Mersenne Twister''
(algo así como el torcedor de Mersenne) que usa los números primos de Mersenne,
es decir, del tipo $2^{2^{n-1}}$. La secuencia que se genera es periódica, pero
el periodo es de $2^{19937}-1 \simeq 10^{600}$ números. 

Para iniciar la
secuencia de números aleatorios, se utiliza una \defn{semilla} (``seed''), que
es simplemente el 
valor inicial con el cual se empieza la iteración. Al utilizar otra semilla, se
generará otra secuencia de números pseudo-aleatorios.




% De hecho, una maquina determinista \emph{no} puede generar números realmente
% \emph{aleatorios}, sino nada más \emph{pseudo}-aleatorios. Éstos se ven
% ``como si fueran'' realmente aleatorios, es decir, tienen propiedades
% estadísticas que se aproximan a las de una serie de números aleatorios
% verdadero. Estos números pseudo-aleatorios se generan a partir de una
% iteración
% \emph{determinista}, pero que da como resultado una secuencia que parece ser
% aleatorio.


\section{Funciones}

Las \emph{funciones} se pueden considerar como subprogramas, o subrutinas,  que
ejecutan una tarea reducida. Cada función debería de corresponder a una tarea
dada.

Las funciones se \emph{declaran} como es el caso de la función \inl!main!. El
caso más sencillo es:
\begin{lstlisting}
void saludo() {
  cout << "Bienvenido al programa." << endl;
}
\end{lstlisting}
Aquí:
\begin{itemize}
\item \inl{f} es el nombre de la función
\item \inl{()} es la lista de parámetros de la función
\item el bloque \inl!{! $\cdots$ \inl!}! es el cuerpo de la función --las
operaciones que la función llevará a cabo
\item \inl!void! quiere decir que la función no regresa ninguna información.
\end{itemize}

El declarar una función \emph{no implica} que la función se utilizará. 
Para utilizar una función, hay que \defn{llamarla} desde otro lado del programa.
Para hacerlo, se pone el nombre de la función, seguido por paréntesis que
contienen los argumentos que se enviarán a la función, por ejemplo:
\begin{lstlisting}
int cuad(int a) {
  return a * a;
}

int main() {
  int a = 17;
  cout << "El cuadrado de " << a << " es " << cuad(a);
}
\end{lstlisting}
Esta función acepta un \defn{argumento} \inl{a}, cuyo tipo debemos especificar.
También regresa un entero.

Puede haber mas que un \inl{return} en una función.

Una función debe ser corta,
tener un nombre que representa lo que hace, y ser clara.
Una vez que se ha comprobado que una función hace lo que debería de
hacer, se puede olvidar del contenido de la función y reutilizar la misma
dentro del mismo u otros programas.  Si una función crece demasiado, 
es necesario dividirlo en distintas funciones, cada una de las
cuales hace una tarea propia.  Estas sub-funciones se pueden llamar desde la
función principal.

De la misma manera, la función \inl{main} \emph{no} debería contener todo el
programa, sino llamar a distintas funciones con nombres representativos,
quienes llevan al cabo las distintas tareas requeridas.

\ejercicio 
Calcular e imprimir las raíces cuadradas y las raíces cuárticas de los enteros
de $1$ a $100$ utilizando el algoritmo Babilónico.


\section{Rutinas para números aleatorios}

En C++, ya existe una función \inl{rand()} para generar números aleatorios
\emph{enteros}, que es suficiente para un uso ``casual''.
% \footnote{No es adecuado para un uso
% más serio, para lo cual se puede emplear, por ejemplo, el algoritmo llamado 
% \defn{Mersenne Twister}.}:
Esta función está declarada en la librería \inl{cstdlib}.
Esta función
regresa un número aleatorio entero entre $0$ y una constante muy 
grande \inl{RAND_MAX},  que normalmente está dado por de $2^{31} - 1 \simeq 2
\times 10^9$.
(También viene definido en \inl{cstdlib}.)

Para generar un número real en el intervalo $[0,1)$, uno pensaría en
\begin{lstlisting}
double drand() { 
  return rand() / RAND_MAX;
}
\end{lstlisting}
Sin embargo, aquí estamos dividiendo dos números enteros, lo cual da $0$ (o tal
vez $1$ de vez en cuando), ya que operaciones entre enteros siempre regresan la
parte entera de la respuesta. Por lo tanto, es necesario convertir uno de los
enteros a precision doble:
\begin{lstlisting}
double drand() {
	return double(rand()) / RAND_MAX;
}
\end{lstlisting}
Esto está perfecto para números en el intervalo $[0,1]$. Para el intervalo
$[0,1)$ (es decir, que no incluye $1$), 
hay que usar
\begin{lstlisting}
double drand() {
	return double(rand()) / (RAND_MAX+1.);
}
\end{lstlisting}

\ejercicio
\begin{itemize}
\item >Cómo se pueden generar números reales en el intervalo $[a,b)$?
\item >Qué tal números enteros en $[i,j)$?
\item >Cómo se pueden generar $-1$ ó $1$ con probabilidad $\textfrac{1}{2}$
cada uno?
\item >Cómo se puede generar $0$, $1$, $2$ ó $3$ con probabilidades $0.5$,
$0.25$, $0.125$ y $0.125$, respectivamente?
\end{itemize}

\section{Archivos de cabecera}

Ya que tenemos el código para generar distintos tipos de números aleatorios, lo
podemos colocar en un \defn{archivo de cabecera}, llamado, por ejemplo,
\inl{alea.h}.  En cada programa que utiliza este mismo código,  incluimos
la línea
\begin{lstlisting}
#include "alea.h"
\end{lstlisting}
y entonces las funciones ahí definidas estarán disponibles.
En general, los archivos de cabecera proveen código útil que se puede incluir
en distintos programas para reutilizarse, sin tener que reescribirse cada vez.



Nótese que en C++, se puede ocupar el mismo nombre para dos funciones
distintas, siempre y cuando sus argumentos sean distintos.


\ejercicio
¿Cómo se podría generar ``A'' con $\frac{1}{3}$ ó ``B'' con
$\frac{2}{3}$?

\ejercicio
Generar ``A'' con probabilidad $\frac{1}{3}$ y  ``B'' con
probabilidad $\frac{2}{3}$ un total de $N$ veces. Contar el número de As y Bs
que se  obtienen realmente.


\section{Simulación de una caminata aleatoria en una dimensión}

Con las herramientas que hemos desarrollado hasta la fecha, no está difícil
implementar la simulación de una caminata aleatoria en una red uni-dimensional.

La única pieza de información que el caminante requiere es su posición, que
será un número entero.  En cada paso, se escoge si brincará a la izquierda o a
la derecha, o si se quedará en el mismo lugar, y se actualiza su posición de
acuerdo con esta elección.

\ejercicio 
Implementa una caminata aleatoria discreta en una dimensión en un programa
llamado \inl{caminata}.
Dibujar su posición contra el tiempo.

\section{Correr varias veces una simulación}
Si tenemos una simulación de este tipo, >cómo podríamos correrlo varias veces y
dibujar la salida de todas las corridas? 

Una manera sería implementar en el mismo programa otro bucle, donde se
especifica el número de veces que correrse. Pero otra manera, que puede ser más
flexible, es implementar el bucle desde la línea de comandos. Utilizaremos
\inl{bash} para los ejemplos --la sintaxis cambia para distintos tipos de
shell. (Para entrar en una sesión de \inl{bash}, si se encuentra utilizando
otro shell, simplemente se pone el comando \inl{bash}.)

Muchos de los comandos que teclean en la línea de comandos son programas que 
\inl{bash} corre. Uno de ellos en Linux es \inl{seq}, que genera
secuencias de números\footnote{En BSD (utilizado en MacOS), el equivalente es
\inl{jot 10 1}.}:
\begin{lstlisting}
$ seq 1 10
\end{lstlisting}
\inl{bash} provee una manera de captar la salida de este comando con la
sintaxis
\begin{lstlisting}
$ echo $(seq 1 10)
\end{lstlisting}
donde el comando \inl{echo} simplemente imprime su argumento y
\inl{$(} $\cdots$ \inl{)} capta la salida del comando.

Ahora podemos \defn{iterar} sobre una lista de palabras con un \inl{for}:
\begin{lstlisting}
$ for i in $(seq 1 10); do echo "Hola"; done
\end{lstlisting}
El valor de una variable se obtiene a través de \inl!$!:
\begin{lstlisting}
$ for i in $(seq 1 10); do echo $i; done
\end{lstlisting}

\ejercicio
Compara la salida de la siguientes dos comandos:
\begin{lstlisting}
$ for i in $(seq 1 10); do echo $i; done
$ for i in seq 1 10; do echo $i; done
\end{lstlisting}

\ejercicio
Corre el programa \inl{caminata} 100 veces y captar su salida a un archivo.
Grafica la posición contra el tiempo para cara corrida.
Para eso, es útil saber que \inl{gnuplot} trata bloques de líneas de datos en un
archivo como corridas distintas si están separadas por dos líneas en blanco.
En la versión de \inl{gnuplot} 4.3 y mayores\footnote{El comando \inl{gnuplot
--version} indica qué versión es.}, se puede iterar en estos distintos bloques
(llamados \defn{índices}) del archivo con la siguiente notación:
\begin{lstlisting}
plot for [j=1:100] "caminata.dat" index j	# abreviacion de index: i
\end{lstlisting}


% 
% 
% \section{Colecciones de datos: arreglos}
% 
% >Cómo podríamos simular una caminata aleatoria en una red cuadrada en dos
% dimensiones? En cada paso, el caminante debería escoger una dirección al azar,
% por ejemplo con igual probabilidad.
% 
% La posición en $d$ dimensiones espaciales está representada por $d$
% coordenadas.
% La manera más sencilla de representar este vector es simplemente definiendo
% $d$ 
% variables enteras por separado, por ejemplo, \inl{x} y \inl{y} en dos
% dimensiones. Sin embargo, esta manera no refleja de manera muy fiel la
% estructura matemática; tampoco será muy fácil de extender a otras dimensiones.
% 
% Por lo tanto, es necesario buscar una \defn{estructura de datos} capaz de
% representar a este conjunto de coordenadas. Matemáticamente, podemos pensar en
% un \defn{vector} con $d$ entradas; buscamos entonces el análogo.
% 
% El análago es un \defn{arreglo} --una estructura que puede contener $d$ datos
% del mismo tipo. En C++, la manera más básica de declarar un arreglo de $d$
% entradas es
% \begin{lstlisting}
% int posicion[2];
% \end{lstlisting}
% Aquí, \inl{posicion} se declara como un arreglo de 2 enteros, los cuales se
% pueden accesar con
% \begin{lstlisting}
% posicion[0] = 3;
% posicion[1] = -17;
% cout << "El vector es (" << posicion[0] << ", " << posicion[1] << ")\n"
% \end{lstlisting}
% Nótese que la numeración empieza en \defn{cero} y termina en $d-1$ en C++.
% 
% \section{Contenedores tipo \inl!vector!}
% 
% Sin embargo, esta manera de declarar  arreglos resulta ser poco flexible.
% Otra manera más flexible, disponible solamente en C++, es utilizando
% la librería \inl{vector}, que forma parte de la biblioteca estándar de C++. 
% \inl{vector} es un \defn{contenedor}, que es cualquier estructura de datos que
% contiene distintos datos del mismo tipo. Se declara como sigue:
% \begin{lstlisting}
% vector<int> posicion(2);
% \end{lstlisting}
% lo cual dice que posicion es un \inl{vector} que contiene enteros, y tiene
% tamaño inicial 2 (el número de elementos que puede contener).
% La sintaxis para accesar los elementos de un vector es igual a la para accesar
% un arreglo: \inl{posicion[0]}, etc.  
% 
% La ventaja de un \inl{vector} es que su tamaño puede cambiar mientras el
% programa corre. Por ejemplo, para agregar un elemento al final del
% \inl{vector}, se utiliza la función \inl{push_back}:
% \begin{lstlisting}
% posicion.push_back(5);
% cout << "Posicion tiene " << posicion.size() << " elementos\n";
% \end{lstlisting}
% 
% Aquí, \inl{push_back()} y \inl{size()} (que regresa el número de elementos
% que están actualmente adentro del \inl{vector} son funciones. El punto entre
% \inl{posicion} y los nombres de estas funciones indica que son funciones que
% ``le pertenecen'' al \defn{objeto} \inl{posicion}. Este mecanismo permite que
% distintos tipos de objetos tengan funciones con el mismo nombre.
% 
% 
% Nótese que el nombre \inl{vector} es un poco confuso del punto de vista
% matemática, ya que actúa simplemente como un contenedor de datos --no hay
% ninguna operación matemática definida para este tipo de \inl{vector}.
% Visto sus capacidad de cambiar de tamaño, podemos pensar en \inl{vector} más
% como una lista lineal de objetos de un tipo dado que un vector matemático.
% 
% \section{Caminatas aleatorias en dos dimensiones}
% Ahora podemos regresar a implementar una caminata aleatoria en dos
% dimensiones.
% 
% \ejercicio\\
% (i) Implementa una caminata aleatoria en 2 dimensiones, con una posición que
% es
% un vector de dos coordenadas. En cada paso, se elige una dimensión al azar, y
% se
% elige al azar si incrementar o decrementar la coordenada en esta dimensión.
% 
% (ii) Traza la trayectoria de una caminata aleatoria en 2 dimensiones.
% Utiliza el comando \\
% \inl{set size ratio -1} para que los ejes se vean en la
% proporción correcta (1:1).
% 
% (iii) Traza la trayectoria de 100 caminatas aleatorias.
% 
% (iv) Dibuja las posiciones \emph{finales} de 10000 caminatas aleatorias
% después
% de distintos números de pasos.
% 
% \ejercicio
% Repite el ejercicio anterior para una caminata aleatoria en 3 dimensiones,
% utilizando el comando \inl{splot} de \inl{gnuplot} para dibujar en 3
% dimensiones. (En las nuevas versiones de \inl{gnuplot}, se puede utilizar
% \inl{set view equal xyz} para obtener un efecto parecido a 
% \inl{set size ratio -1} en dos dimensiones.)

\section{Cambiar los números aleatorios}

Si volvemos a correr un programa que utiliza los números aleatorios que provee
\inl{drand()}, encontraremos que los mismos números salen en cada corrida del
programa.  Eso puede ser útil --por ejemplo, para identificar dónde ocurre un
error.

Sin embargo, normalmente querramos generar distintos números aleatorios en cada
corrida. Para hacerlo, basta con incluir la línea
\begin{lstlisting}
#include <ctime>

int main() {
  srand(time(0));
}
\end{lstlisting}
La función \inl{srand()} inicializa la \defn{semilla} de los números aleatorios
con su argumento. En este caso, utilizamos el tiempo del sistema, dado por
\inl{time(0)}, para escoger una semilla distinta cada vez\footnote{Esta
función regresa el tiempo en segundos desde el 1ero de enero de 1970. Por lo
tanto, dará el mismo resultado si se corre un mismo programa varias veces
antes de que se haya actualizado este número de segundos.}.

