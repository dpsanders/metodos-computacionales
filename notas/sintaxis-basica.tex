\chapter{Sintaxis básica de C++}

% \source{progs/primero.cpp}{}

C++ es un lenguaje de programación \emph{compilado} --es decir, para correr un
programa, es necesario pasar a través de una etapa de compilación para
convertirlo en código que la máquina puede utilizar. Aún si eso puede ser
latoso, hace más eficiente (rápido) el procedimiento final de correr el
programa. 

C++ se desarrolló a partir del lenguaje C, por lo cual retiene la sintaxis
básica del mismo. Sin embargo, C++ tiene conceptos de más alto nivel que C, 
los cuales
adoptaremos  desde el principio.

% \lstinputlisting{progs/primero.cpp}

\section{Primeros pasos}
El primer programa clásico es el ``<Hola, mundo!'', que en C++ se podría
escribir como sigue:
\source{primero.cpp}{}

Para correrlo, guardamos el código en el archivo \ttt{primero.cpp} y lo
compilamos y corremos con\footnote{El signo \inl{\$} denota el ``prompt''
en la terminal donde se teclean los comandos}:
\begin{lstlisting}
$ g++ primero.cpp -o primero
$ ./primero
Hola, mundo!
\end{lstlisting}
% }

Aquí, \inl!g++! es el nombre del compilador de C++ del proyecto GNU, y \inl!-o!
indica que la salida (``output'') va a ser un ejecutable que se llama
\inl!primero!. El comando \inl!./primero! ejecuta este programa --el punto
indica que se encuentra en el directorio actual.

En el código, el \lstinline!#include! es una instrucción al compilador de
\defn{incluir} un archivo de código fuente predefinido, que en este caso provee
los comandos \inl!cout! y \inl!endl!. El primero imprime lo que sigue después
del \inl!<<!, y el segundo da una nueva línea.  Los dos están adentro del
\defn{espacio de nombres} \inl!std!  (``estándar''). Como se vuelve molesto
teclear \inl!std::! todo el tiempo, es conveniente utilizar el comando
\inl!using namespace std!, que da la siguiente versión mejorada del código:

\source{primero2.cpp}{}

La línea \inl!int main()! declara una \defn{función} (subrutina) llamada
\inl!main!. Cada programa de C++ debe contener una función con este nombre,
donde la ejecución del programa empezará. Las parentesis contienen los
argumentos de la función (información que se manda a
la función), e \inl!int! indica que regresa un entero, que indicará si el
programa terminó con exito (valor $0$), como indica \inl!return 0!, o no (valor
no-cero).

Las llaves \inl!{! y \inl!}! encierran un \defn{bloque} de código, que se trata
como una unidad. En este caso, el bloque es el \defn{cuerpo} de la función.

Nótese que cada instrucción, o comando, en C++, tiene que terminar con un
\emph{punto y coma}, \inl!;! --el olvidar algun \inl!;! llevará a un sinfin de
errores al momento de compilar el programa!

Los \defn{comentarios} se ponen después de \inl!//!; el resto de la línea es un
comentario.

El \inl!'\n'! es un carácter especial, que también quiere decir que imprima una
nueva línea; otro parecido es un ``tabulador'' (un espacio especial para
alinear), dado por \inl!'\t'!.  (Los caracteres individuales se escriben entre
apóstrofes, \inl!'!, mientras que las cadenas de varios caracteres se escriben
entre comillas, \inl!"!.)

En el resto de este capítulo, normalmente daremos fragmentos de código. Para
correrlos, es necesario incluirlos en un programa completo de este tipo.

\section{Variables}
Para llevar a cabo cálculos, necesitamos poder guardar datos en
\defn{variables}. Éstas se se tienen que \defn{declarar}, es
decir, especificar de qué tipo son, 
antes de utilizarse:

\begin{lstlisting}
int num_particulas;   	  // declarar una variable a de tipo entero (integer)
double temperatura;   // declarar una variable real de doble precision
\end{lstlisting}

Una declaración  reserva suficiente espacio en la memoria
para guardar el valor de la variable, y le da información al compilador
que puede utilizar para detectar errores, si intentamos manipular una variable
de una manera que no corresponde a su tipo.

Principalmente usaremos \inl!int! y \inl!double!.  Normalmente ya no se utilizan
los números reales de precisión sencilla (\inl!float!)\footnote{Eso ya ha
cambiado con la actual programación de los GPUs, que están diseñados para
funcionar en precisión sencilla.}, ya que
las máquinas modernas están optimizadas para usar números de doble precisión.

Se asignan valores a las variables con \inl{=}, que también se puede hacer al
momento de declarar las variables:
\begin{lstlisting}
num_particulas = 10;
temperatura = 1.5;

int a = 3;
double b = -1.3e5;    // notacion cientifica 
\end{lstlisting}

Los valores de las variables, como la mayoría de los objetos, se pueden imprimir
utilizando \inl!cout!, como sigue. Lo que va dentro de comillas (\inl!"!) se
imprime exactamente tal cual. Los \inl!<<! se pueden encadenar el número de
veces que sea necesario (siempre respetando que el código sea legible):
\begin{lstlisting}
cout << "a = " << a << "; b = " << b << endl;
\end{lstlisting}

Las declaraciones de variables se pueden hacer en \emph{cualquier lugar} del
programa\footnote{A diferencia de C.}.  El mejor estilo es el de declarar una
variable lo más cercano a su primer uso.


\section{Flujo de entrada}

Para permitir al usuario que introduzca datos para guardarse en variables,
se puede utilizar \inl!cin!, de manera opuesta a \inl!cout!:
\begin{lstlisting}
int num_particulas;
cout << "Cuantas particulas? ";
cin >> num_particulas;

double temperatura, presion;
cout << "Valor de temperatura y presion? ";
cin >> temperatura >> presion;
cout << "Usando temperatura = " << temperatura 
		<<  "; presion = " << presion << endl;
\end{lstlisting}
(Puede causar problemas el utilizar acentos dentro del código, por lo cual los
evitamos.)
Nótese que en este caso, si la entrada que damos es \inl!3, 4!, entonces el
valor de presión está equivocada. En un programa real, habría que checar este
tipo de errores.



\section{Condicionales}

Muy a menudo, es necesario comprobar si alguna condición se satisfaga o no.
El caso más sencillo se implementa con \inl!if!:
\begin{lstlisting}
int a = 3; 
if (a > 0)  {      // NB: Las parentesis son obligatorias
  cout << "a es positivo" << endl; 
}
\end{lstlisting}
Nótese que se utiliza un bloque, entre \inl!{! y \inl!}!, para delimitar el
código que se ejecutará si la condición se satisface.
Se puede agregar una alternativa que se ejecuta si la condición no se satisface:
\begin{lstlisting}
else {
  cout << "a es negativo " << endl;
  cout << "Su valor absoluto es " << -a << endl;
}
\end{lstlisting}
Nótese que aquí hay dos comandos que se ejecutan como parte del bloque.

Las condiciones que se pueden probar incluyen:
\begin{center}
\begin{tabular}{|l|c|}
\hline
igualdad & \verb!==! \\
desigualdad & \verb/!=/ \\
AND & \verb!&&! \\
OR & \verb!||! \\
NOT & \verb/!/ \\
mayor que & \verb!>! \\
mayor o igual que & \verb!>=! \\
menor que & \verb!<! \\
menor o igual que & \verb!<=! \\
\hline 
\end{tabular}
\end{center}

Para probar más de una condición simultáneamente, se utilizan \inl!&&!  
(y) y \inl!||! (ó).   Las condiciones que se juntan se ponen dentro de
otra paréntesis:

\begin{lstlisting}
int a = 3, b = 5;

if ( (a < 0) && (b < 0) ) {
	cout << "a y b son negativos los dos" << endl;
}

if ( (a < 0) || (b < 0) ) {
	cout << "Al menos uno de a y b es negativo" << endl;
}
\end{lstlisting}



\section{Bucles}

El poder de las computadoras para llevar a cabo cálculos proviene de su
capacidad de repetir las mismas operaciones una y otra vez, a través de los
\defn{bucles}.

\subsection{\inl!while!}

El bucle más sencillo, y más básico, es el \inl{while}, que se ejecuta
\emph{mientras} una
condición se satisface.  Normalmente se utiliza un \inl{while} cuando no se
sabe de antemano
cuantas veces hay que iterar.  Las condiciones se emplean de la misma manera
como para un \inl{if}.

Por ejemplo, una manera (<no necesariamente la mejor!) de contar la mayor
potencia de $2$ menor que un número dado sería:
\begin{lstlisting}
int valor = 63201;
int potencia = 0;

while (valor > 1) {
  cout << "valor = " << valor << endl;
  valor /= 2;     // equivalente a:  valor = valor / 2
  potencia ++;
}

cout << "La mayor potencia de 2 es " << potencia+1 << endl;
\end{lstlisting}
Nótese que la aritmética con variables se lleva a cabo de manera intuitiva.
Es un procedimiento muy común el actualizar a una variable, reemplazando su
valor actual con un valor nuevo que se calcula mediante el valor actual.

Para este fin, C++ provee una serie de operadores muy prácticos, que combinan
un operador aritmético con un \inl{=}. Por ejemplo, \inl{a+=b} es una
abreviación de \inl{a = a+b}. Estas combinaciones tienen la ventaja de
enfatizar el hecho de que se está actualizando el valor de una variable a partir
del valor actual.

Además, hay operadores especiales \inl{++} y \inl{--} para incrementar y
decrementar, respectivamente, el valor de una variable \emph{entera} solamente.


\subsection{\inl!for!}

Normalmente se utiliza un \inl{for} cuando se sabe de antemano cuantas
veces se quiere iterar.  La sintaxis es:
\begin{lstlisting}
for (inicializacion;  condicion;  actualizacion) {
  ...
}
\end{lstlisting}
donde
\begin{itemize}
\item inicialización: lo que hace al principio del bucle;
\item condición: lo que se prueba al final de cada iteración;
\item actualización: el comando que se ejecuta al final de cada iteración.
\end{itemize}

Por ejemplo, para sumar los números de $1$ a $10$:
\begin{lstlisting}
int n = 10;
int total = 0;

for (int i = 0; i < n; i ++) {
	cout << i << "\t" << total << endl;    // '\t' es un caracter de tab
	total += i;  
}
cout << "El total es " << total << endl;
\end{lstlisting}

Nótese que se declaró el entero \inl{i} en el \emph{paso de inicializacion} del
bucle. Por lo tanto,  esta variable existe \emph{solamente dentro del
bucle}.  Así se pueden reutilizar nombres de variables comunes, sin
tener que inventar nuevos nombres.

\subsection{\inl!do!--\inl!while!}

Un bucle \inl{do}--\inl{while} es el opuesto de un \inl{while}: la condición se
evalúa al final del bucle, así que el código se ejecuta al menos una
vez. Se utiliza tal vez menos que los otros dos tipos de bucle.
\begin{lstlisting}
int i = 0;
do {
	cout << i << endl;
	i++;
} while (i < 10);
\end{lstlisting}


 \section{Funciones matemáticas}


Algunas funciones matemáticas ya están  implementadas. Para utilizarlas, se
requiere incluir la librería \inl!cmath!:
\begin{lstlisting}
#include <cmath>

cout << "El seno de 3.5 es " << sin(3.5);
cout << "La raiz cuadrada de 700 es " << sqrt(700);
\end{lstlisting}

La librería \inl{cmath} incluye, entre otros, \inl{exp}, \inl{sin}, \inl{cos},
\inl{tan}, \inl{asin} 
  (seno inverso o arcseno), \inl{acos}, \inl{atan} y, \inl{log}. También hay
otras funciones 
evidentes, por ejemplo \inl{log1p(x)} regresa $\log(1+x)$. Para tomar potencias
existe
\inl{pow(x,y)}, que devuelve el valor de $x^y$.

Todas las funciones toman como argumentos números tipo \inl{double}.

Para funciones más complicadas, por ejemplo funciones especiales, hay librerías
disponibles como software libre. Una muy buena es el Gnu Scientific Library
(GSL). Sin embargo, hay que reconocer que el interfaz de esta librería es de
estilo C, no C++ propiamente.

\ejercicio
Implementa el \emph{algoritmo Babilónico} para encontrar la raíz cuadrada de
un numero $y$, como sigue.
Si $a_n$ es un estimado de la raíz cuadrada, entonces un mejor estimado es
\begin{equation}
 a_{n+1} := \frac{1}{2} \left(a_n + \frac{1}{a_n} \right).
\end{equation}
Pon esto adentro de una \defn{iteración}, empezando desde cualquier condición
inicial $a_0 \neq 0$.
Compara el resultado obtenido a cada paso con
el
resultado exacto. Utiliza una gráfica para investigar la velocidad de
convergencia del algoritmo.



\section{Resguardo de datos}
Ahora que hemos producido algunos datos, debemos procesarlos para entenderlos.
La primera etapa es simplemente graficarlos para tener una idea intuitiva de la
forma de los mismos.

Para poder hacerlo, hay dos maneras posibles: (i) mandar los datos a graficar
directamente desde el mismo programa de C++; y (ii) guardar los datos en un
archivo, que luego se graficará con otra herramienta. Posteriormente
abarcaremos el inciso (i); por el momento, ocuparemos la herramienta
\texttt{gnuplot} para graficar los datos.

Una opción para guardar los datos en un archivo es la de abrir un archivo
dentro del programa y escribir ahí. Esta opción la veremos más adelante; por el
momento, aprovecharemos un método que nos proporciona el \defn{shell}, o
\defn{terminal} en Unix: podemos \defn{redirigir} la salida de cualquier
programa a un archivo.  

Por ejemplo, para mandar la salida del programa
\inl{babilonico} a un archivo llamado \inl{sqrt.dat}, podemos poner
directamente en la línea de comandos el comando
\begin{lstlisting}
$ babilonico > sqrt.dat
\end{lstlisting}
Ya no vemos la salida del programa en la pantalla, sino se guarda la misma en
el archivo. En este caso, sobreescribirá el contenido del archivo. Para agregar
los nuevos datos al contenido ya existente, hacemos
\begin{lstlisting}
$ babilonico >> sqrt.dat
\end{lstlisting}

\section{Ver contenido de un archivo}

Para verificar que los datos se han guardado de manera adecuada, podemos copiar
el contenido del archivo a la terminal con el comando
\begin{lstlisting}
$ cat sqrt.dat
\end{lstlisting}
El comando \inl{cat} tiene como fin el de ``concatenar'' su entrada a su salida.
En este caso, la entrada es el archivo \inl{sqrt.dat}, y la salida es la
terminal.

Otro uso de \inl{cat} es para crear un archivo, por ejemplo
\begin{lstlisting}
$ cat > datos1.dat
\end{lstlisting}
creará un archivo \inl{datos1.dat} con el contenido que se teclea en la
terminal --aquí, la entrada es la terminal y la salida se ha redireccionado al
archivo.

Finalmente, \inl{cat} puede literalmente concatenar archivos, por ejemplo
\begin{lstlisting}
$ cat datos1.dat datos2.dat >> datos3.dat
\end{lstlisting}

\section{Graficamiento de datos con \inl{gnuplot}}
Ahora contamos con los datos en un archivo de texto \inl{sqrt.dat}.
Utilizaremos \texttt{gnuplot} para graficarlos.

\texttt{gnuplot} es un programa con el que podemos producir gráficas
en $2$ y $3$
dimensiones, con calidad de publicación (<si trabajamos un poco!)
 Posee una interfaz de línea de comandos, y
también puede ser utilizado desde otro programa o al escribir ``scripts'' 
(programas con listas de comandos de \texttt{gnuplot}).
Cuenta con gráficas interactivas, además de 
una gran variedad de formatos para exportar las gráficas producidas, 
incluyendo a Postscript y \texttt{png}.
Hay distintas demostraciones de sus capabilidades en la página
\url{http://www.gnuplot.info/demo/}.

En la línea de comandos ejecutamos
\begin{lstlisting}
 $ gnuplot
\end{lstlisting}
y así entramos en el entorno del programa.



\subsection{Gráficas de funciones}
Las gráficas más sencillas son las de funciones predefinidas, como
\begin{lstlisting}
$ gnuplot
gnuplot> plot sin(x)
gnuplot> plot sin(x), cos(x) 	# graficar dos funciones juntas
\end{lstlisting}

Para dibujar estas funciones, las evalua en distintos puntos
\begin{lstlisting}
gnuplot> plot sin(x) with points 		# abreviacion: w p
gnuplot> set samp 500 		#aumentar el numero de puntos
gnuplot> replot 		# o teclear 'e' en la ventna de la grafica
\end{lstlisting}

Se pueden definir variables y funciones 
\begin{lstlisting}
gnuplot> a=3
gnuplot> plot sin(a*x)
gnuplot> print a
3
\end{lstlisting}

\subsection{Dibujar datos desde archivos}

Para dibujar el contenido del archivo \inl{sqrt.dat}, ponemos
\begin{lstlisting}
gnuplot> plot "sqrt.dat" 		# usa solo la primera (x) y segunda columna (y)
\end{lstlisting}
Por defecto, dibuja utilizando puntos.
\begin{lstlisting}
gnuplot> plot "sqrt.dat"  with lines		# abreviacion: w l
gnuplot> plot "sqrt.dat", "" using 2:3  	# abreviacion: u
\end{lstlisting}
Aquí, dibujamos dos veces el mismo archivo (al poner un nombre de archivo en
vacío), y ocupamos las columnas $2$ y $3$ para la segunda gráfica.
También se pueden mezclar datos y funciones:
\begin{lstlisting}
gnuplot> plot "sqrt.dat"  w points, sqrt(x) w l		# abreviacion: w p
\end{lstlisting}
% 
% 
% 
% gnuplot> plot "in.dat" using 2:3 #usar las columnas 2 y 3 del archivo
% gnuplot> plot "in.dat", "in.dat" using 1:3 #usar las columnas 1 y 3 
% gnuplot> plot "in.dat", '' u 2:3   #gráfica (1,2) y (2,3)
%                                    # no repetir el nombre del archivo
% gnuplot> plot "in.dat", x*x #mezcla de datos y funciones
% gnuplot> plot "in.dat" u 3:($1+$1)   #se pueden manipular los datos
% #eje x es la columna 3 y el eje y la suma de la columna 1 (1+1) 
% gnuplot> plot "in.dat" u ($1+$2):(log($3)) #dibuja x^3 en función de 
%                                            #x+x^2
% \end{Verbatim}
% 
% \begin{figure}[h!]
% \includegraphics[width=0.5\linewidth]{2-14feb2008/gnuplot/data1.eps}
% \includegraphics[width=0.5\linewidth]{2-14feb2008/gnuplot/data2.eps}
% \caption []{\label{data12}}
% \end{figure}
% 
% \begin{figure}[h!]
% \includegraphics[width=0.5\linewidth]{2-14feb2008/gnuplot/data3.eps}
% \includegraphics[width=0.5\linewidth]{2-14feb2008/gnuplot/data4.eps}
% \caption []{\label{data34}}
% \end{figure}
% 
% \begin{figure}[h!]
% \includegraphics[width=0.5\linewidth]{2-14feb2008/gnuplot/data5.eps}
% \includegraphics[width=0.5\linewidth]{2-14feb2008/gnuplot/data6.eps}
% \caption []{\label{data56}}
% \end{figure}
% 
% Entramos e




\subsection{Exportación de las gráficas}
\ttt{gnuplot} cuenta con un sistema de ayuda. Por ejemplo, podemos conocer
las distintas ``terminales''  (formatos de salida) disponibles con
\begin{lstlisting}
gnuplot> help set term
\end{lstlisting}

Para crear un archivo de Postscript (formato EPS) con la grafica, que cuente
con etiquetas en los ejes, tenemos:
\begin{lstlisting}
gnuplot> set xlabel "x"
gnuplot> set ylabel "sin(3x)"
gnuplot> set term post eps
gnuplot> set out "sin.eps"
gnuplot> replot
gnuplot> set out
gnuplot> !okular sin.eps		# manda un comando a la terminal
\end{lstlisting}

Un ejemplo interesante:
\begin{lstlisting}
gnuplot> f(x) = sin(a*x)
gnuplot> plot a=1, f(x) title 'sin x', a=2, f(x) title 'sin 2x', 
              a=3, f(x) t 'sin 3x'
\end{lstlisting}

% 
% \begin{figure}[h!]
% \includegraphics[width=0.5\linewidth]{2-14feb2008/gnuplot/sin.eps}
% \includegraphics[width=0.5\linewidth]{2-14feb2008/gnuplot/sincos.eps}
% \caption []{\label{sincos} }
% \end{figure}
% \begin{figure}[h!]
% \includegraphics[width=0.5\linewidth]{2-14feb2008/gnuplot/points.eps}
% \includegraphics[width=0.5\linewidth]{2-14feb2008/gnuplot/morepoints.eps}
% \caption []{\label{morepoints} }
% \end{figure}
% \begin{figure}[h!]
% \includegraphics[width=0.5\linewidth]{2-14feb2008/gnuplot/variables.eps}
% \includegraphics[width=0.5\linewidth]{2-14feb2008/gnuplot/function.eps}
% \caption []{\label{functions} }
% \end{figure}

 
