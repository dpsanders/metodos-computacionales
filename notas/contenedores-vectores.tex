\chapter{Contenedores de datos: vectores}

Muy a menudo es importante poder guardar y manipular distintas variables que
están relacionadas entre sí de una u otra manera.

\section{Caminante en dos dimensiones}

>Cómo podríamos simular una caminata aleatoria en una red cuadrada en dos
dimensiones? En cada paso, el caminante debería escoger una dirección al azar,
por ejemplo con igual probabilidad.

La manera más fácil de hacerlo es extendiendo el método de un caminante en una
dimensión, dividiendo el intervalo $[0,1)$ ahora en cuatro partes, cada una de
las cuales corresponde a una dirección diferente, y hacer un \inl{if} con
varias opciones:
\begin{lstlisting}
int x, y;	# posicion de caminante en dos dimensiones
double r = drand();

if (r < 0.25) {
  x++;
}
if ( (r > 0.25) && (r < 0.5) ) {
  y++;
}
if ( (r > 0.5) && (r < 0.75) ) {
  x--;
}
if ( (r > 0.75) ) {
  y--;
}
\end{lstlisting}
Nótese que la única manera de expresar condicionales del estilo de $0.25 < x <
0.5$ en C++ es a través de una condicional de la forma \inl{&&}.

Una manera más corta de expresar lo mismo es utilizando \inl{else}, seguido por
otro \inl{if}:
\begin{lstlisting}
if (r < 0.25) {
  x++;
}
else if (r < 0.5) {
  y++;
}
else if (r < 0.75) {
  x--;
}
else {
  y--;
}
\end{lstlisting}

Sin embargo, estos conjuntos de \inl{if} y \inl{else} son difíciles de leer,
está fácil de cometer un error, y además es muy poco extendible. Por ejemplo,
si ahora queremos estudiar un caminante en $3$ --ó en $d$-- dimensiones,
entonces tenemos que reescribir el código por completo.  Por lo tanto,
necesitamos replantear el problema, para que sea fácilmente extendible a otras
situaciones.


\section{Colecciones de datos: arreglos}

La posición en $d$ dimensiones espaciales está representada por $d$ coordenadas.
La manera más sencilla de representar este vector es simplemente definiendo $d$ 
variables enteras por separado, como hicimos con \inl{x} y \inl{y} en dos
dimensiones. Sin embargo, esto no refleja la
estructura matemática, y tampoco será fácil de extender a otras dimensiones.

Por lo tanto, es necesario buscar una \defn{estructura de datos} capaz de
representar a este conjunto de coordenadas. Matemáticamente, podemos pensar en
un \defn{vector} con $d$ entradas; buscamos entonces el análogo en C++ de un
vector, visto como un conjunto ordenado de números.

Este análogo es un \defn{arreglo} --una estructura que puede contener un número
dado de  datos
del mismo tipo con una orden dada. 
En C++, la manera más clásica --pero menos flexible-- de declarar un
arreglo de $d$
entradas es
\begin{lstlisting}
int posicion[2];
\end{lstlisting}
Aquí, \inl{posicion} se declara como un arreglo de 2 enteros, los cuales se
pueden accesar con
\begin{lstlisting}
posicion[0] = 3;
posicion[1] = -17;
cout << "El vector es (" << posicion[0] << ", " << posicion[1] << ")\n"
\end{lstlisting}
Nótese que la numeración empieza en \defn{cero} y termina en $d-1$ en C++.

\section{Contenedores tipo \inl!vector!}

Sin embargo, esta manera de declarar  arreglos resulta demasiado rígido.
Otra manera más flexible, disponible solamente en C++, es utilizando
la librería \inl{vector}, que forma parte de la biblioteca estándar de C++. 
\inl{vector} es un \defn{contenedor}, que es cualquier estructura de datos que
contiene distintos datos del mismo tipo. Se declara como sigue:
\begin{lstlisting}
vector<int> posicion(2);
\end{lstlisting}
lo cual declara \inl{posicion} como un arreglo que contiene enteros, y
tiene
tamaño inicial 2 (el número de elementos que puede contener).
La sintaxis para accesar los elementos de un vector es igual a la para accesar
un arreglo: \inl{posicion[0]}, etc.  

La ventaja de un \inl{vector} es que es un arreglo \defn{dinámico} --su tamaño
puede cambiar \emph{mientras el
programa corre}. En cualquier momento, podemos utilizar el comando \inl{resize}
para cambiar el número de entradas en el arreglo:
\begin{lstlisting}
vector<int> datos;
cout << datos.size() << endl;

datos.resize(10);
datos.resize(20);
\end{lstlisting}
En la primera línea, se declaró \inl{datos} sin un tamaño. Por lo tanto,
empieza como un arreglo vacío, con cero elementos, como nos muestra el comando
\inl{datos.size()}.

Estas dos funciones son los primeros ejemplos claros que hemos visto de la
\defn{programación orientada a objetos}. Un \inl{vector} es un objeto, que
tiene \defn{propiedades} (variables) --por ejemplo, su tamaño, y los datos que
contiene-- y además tiene acciones (funciones), llamadas \defn{métodos}. Tanto
las propiedades como los métodos le pertenecen al objeto, y se accesan a través
del operador `\inl{.}'. 
Esto tiene la ventaja de que otros objetos pueden tener variables y funciones
con los mismos nombres, sin crear ningún conflicto, ya que queda claro en cada
momento cuál función o variable se requiere, al especificar a qué objeto
pertenece.

Otro método sumamente útil de los \inl{vector}s es \inl{push_back()}, lo cual
agrega un elemento al final del
\inl{vector}. Por lo tanto, no es necesario preocuparse por el tamaño:
\begin{lstlisting}
vector<int> datos;
datos.push_back(5);
cout << "datos tiene " << datos.size() << " elementos\n";
\end{lstlisting}

% Aquí, \inl{push_back()} y \inl{size()} (que regresa el número de elementos
% que están actualmente adentro del \inl{vector} son funciones. El punto entre
% \inl{posicion} y los nombres de estas funciones indica que son funciones que
% ``le pertenecen'' al \defn{objeto} \inl{posicion}. Este mecanismo permite que
% distintos tipos de objetos tengan funciones con el mismo nombre.


Nótese que el nombre \inl{vector} es un poco confuso del punto de vista
matemática, ya que actúa simplemente como un contenedor de datos --no hay
ninguna operación matemática definida para este tipo de \inl{vector}.
Visto sus capacidad de cambiar de tamaño, podemos pensar en \inl{vector} más
como una lista ordenada de objetos de un tipo dado que un vector matemático.

\section{Caminatas aleatorias en dos dimensiones}
Ahora podemos regresar a implementar una caminata aleatoria en dos dimensiones.
La posición será un \inl{vector} de dos coordenadas, las dos enteros. >Cuál es
la dinámica del caminante? 

Visto en coordenadas Cartesianas, el caminante tiene un vector de
desplazamiento $\Delta \xx$ en cada paso, desplazándose de $\xx$ a $\xx' = \xx
+ \Delta \xx$. En el caso más sencillo, este vector en dos dimensiones puede
ser $(1,0)$, $(-1, 0)$, $(0, 1)$ ó $(0,-1)$.  De manera similar, en $5$
dimensiones, tendríamos vectores de desplazamiento del tipo $(0,-1,0,0,0)$. En
cada caso, hay exactamente una coordenada que cambia en cada paso, y esta
coordenada se incrementa o decrementa en $1$.

Por lo tanto, podemos pensar que en cada paso, el caminante decide en qué
dirección moverse: verticalmente u horizontalmente, en el caso de dos
dimensiones.  Luego decide si incrementar o decrementar la coordenada en esta
dirección.  Para eso, necesitamos poder escoger un entero al azar.

\section{Más números aleatorios}
Hasta ahora, hemos generado números aleatorios solamente en el intervalo
$[0,1)$. Para generar números en el intervalo $[0,c)$, para algún valor de
$c>0$, simplemente escalamos por $c$. De ahí podemos ver que podemos generar
números en el intervalo $[a,b)$ haciendo una traslación. Por lo tanto, podemos
declarar una nueva función como sigue:
\begin{lstlisting}
double drand2(double a, double b) {
  return a + (b-a) * drand();
}
\end{lstlisting}
Pasamos dos variables a la función como \defn{argumentos}, cuyos valores
emplea para calcular su respuesta. Además, aprovechamos el hecho de que ya
contamos con la función \inl{drand} para no repetir código.

De hecho, C++ cuenta con un mecanismo de \defn{sobrecarga} de funciones. Eso
quiere decir simplemente que podemos dar el mismo nombre a dos funciones
distintas, siempre y cuando difieren en el número de argumentos que tienen
(para que el compilador puede distinguir cuál se requiere en cada
situación). Por lo tanto, podemos reescribir la declaración como
\begin{lstlisting}
double drand(double a, double b) {
  return a + (b-a) * drand();
}
\end{lstlisting}
Esto tiene la ventaja de que no es necesario recordar cuál nombre hay que
utilizar en cada caso.

Ahora, para generar enteros al azar, podemos asociar el intervalo $[3,4)$ con
$3$, el intervalo $[4,5)$ con $4$, etc. Así que si generamos $3.6$,
regresaremos el entero $3$. Esto se logra con
\begin{lstlisting}
int irand(int i, int j) {
  // Regresa un entero uniforme en $[i, j)$
  // NB: No incluye j como posibilidad

  return int( floor( drand(i, j) ) );
}
\end{lstlisting}
Aquí, dado dos enteros $i$ y $j$, primero generamos un número aleatorio real
entre $i$ y $j$. Nótese que estos enteros se convierten de manera automática a
\inl{double}s al utilizarse como argumentos en la función \inl{drand()}.
Luego se encuentra el mayor entero que es menor o igual al resultado con la
función \inl{floor()}, que está definida en la librería \inl{<cmath>}.
Finalmente, se convierte el tipo de la salida de \inl{floor()} (que regresa un
\inl{double}) a un entero.  Nótese que el número $j$ nunca se puede generar por
esta función. 

\ejercicio Modifica la función \inl{irand()} para generar un entero uniforme
entre $i$ y $j$ que sí puede incluir $j$ como posibilidad.


\section{Caminante aleatorio en 2 dimensiones}
Regresando al caminante aleatorio en 2 dimensiones,
podemos reescribir la parte principal como sigue:
\begin{lstlisting}
vector<int> posicion(2);
int d;

int direccion = irand(0,2); 	

double r = drand();
if (r < 0.5) {
  posicion[direccion] ++;
}
else {
  posicion[direccion] --;
}
\end{lstlisting}

{
%\dbend 
\small
Las condicionales en C++ son expresiones booleanas, que también toman valores.
Por ejemplo, 
\begin{lstlisting}
int b = -1;
int a = (b < 0);
\end{lstlisting}
le asigna a \inl{a} el valor del resultado de evaluar la condición \inl{b < 0}.
Estos valores
se pueden utilizar para hacer cálculos.

\ejercicio >Cómo se puede reescribir el caminante aleatorio en dos dimensiones
% para evitar los \inl{if}? Eso podría ser más eficiente en ciertos tipos de
máquina --como siempre, es cuestión de probar para ver cual opción es más
rápida.
}



% \ejercicio\\
% (i) Implementa una caminata aleatoria en 2 dimensiones, con una posición que
% es
% un vector de dos coordenadas. En cada paso, se elige una dimensión al azar, y
% se
% elige al azar si incrementar o decrementar la coordenada en esta dimensión.

(i) Traza la trayectoria de una caminata aleatoria en 2 dimensiones.
Utiliza el comando \\
\inl{set size ratio -1} para que los ejes se vean en la
proporción correcta (1:1).

(iii) Traza la trayectoria de 100 caminatas aleatorias.

(iv) Dibuja las posiciones \emph{finales} de 10000 caminatas aleatorias después
de distintos números de pasos.

\ejercicio
Repite el ejercicio anterior para una caminata aleatoria en 3 dimensiones,
utilizando el comando \inl{splot} de \inl{gnuplot} para dibujar en 3
dimensiones. (En las nuevas versiones de \inl{gnuplot}, se puede utilizar
\inl{set view equal xyz} para obtener un efecto parecido a 
\inl{set size ratio -1} en dos dimensiones.)

% \section{Mejorando la salida en \inl{gnuplot}}

