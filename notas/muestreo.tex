\chapter{Muestreo}
Como vimos en el capítulo anterior, es normalmente imposible llevar a cabo las
sumas requeridas por la teoría de la física estadística para calcular la
función de partición y los promedios deseados. Por lo tanto, es necesario
\defn{muestrear} ciertas configuraciones ``representativas'', es decir, escoger
de manera estadística o aleatoria --pero al mismo tiempo, lista-- las
configuraciones sobre las cuales sumaremos.

La primera idea que podríamos tener es la de muestrear de manera uniforme sobre
todas las configuraciones, escogiéndolas ``al azar'', es decir, con igual
probabilidad, tal que cada espín tiene igual probabilidad de apuntar hacia
arriba o hacia abajo.  Sin embargo, está intuitivamente claro que eso dará
configuraciones $\ss$ que tienen aproximadamente el mismo número de espines para
arriba como para abajo, y por lo tanto, una magnetización $M(\ss)$ y energía
$E(\ss)$ que se concentran alrededor de $0$. Eso sería adecuado para investigar
las propiedades del sistema a altas temperaturas, donde justamente las
configuraciones tienen un peso estadístico, es decir, probabilidad $p(\ss)$, más
o menos uniforme. Sin embargo, a bajas temperaturas, el sistema se concentrará
alrededor de sus estados bases, donde todos los espines se alínean, mientras
que tales configuraciones casi nunca se generarán de manera uniforme.

\section{Muestreo no-uniforme}
Por lo tanto, es necesario introducir un muestreo no-uniforme, es decir,
escoger distintas configuraciones según una distribución de probabilidad
$p(\mu)$ (que no necesariamente es la de Boltzmann). Más adelante veremos cómo
eso se puede hacer; por lo momento, supongamos que ya lo hemos logrado.

Consideremos un muestreo finito de configuraciones generadas según esta
distribución de probabilidad, $(\mu_1, \ldots, \mu_M)$. En una corrida larga,
esperamos que algunas de estas configuraciones son iguales. Enumeremos las
\emph{distintas} configuraciones como $\mu^{(1)}, \ldots, \mu^{(C)}$. Entonces
$\mu^{(i)}$ debería aparecer $M p(\mu^{(i)})$ veces, aproximadamente. 

Si ahora queremos calcular, por ejemplo, la función de partición, pensaríamos
primero en calcular
\begin{equation}
Z \stackrel{\textrm{?}}{\simeq} \sum_{i=1}^M e^{-\beta E(\mu_i)}.
\end{equation}
Sin embargo, eso daría
\begin{equation}
Z \stackrel{\textrm{?}}{\simeq} \sum_{i=1}^C p(\mu^{(i)}) e^{-\beta
E(\mu^{(i)})}.
\end{equation}
Por lo tanto, las configuraciones aparecen en la suma pesada por la
distribución $p(\mu)$ de muestreo que nosotros imponemos.

Para eliminar este efecto no-deseado, es necesario dividir por las $p$. Así
obtenemos las siguientes expresiones aproximadas para las cantidades de interés:
\begin{equation}
 Z \simeq  \sum_{i=1}^M \frac{1}{p(\mu_i)} e^{-\beta
E(\mu_i)};
\end{equation}
\begin{equation}
 \mean{Q} \simeq \frac{\sum_{i=1}^M \textfrac{1}{p(\mu_i)} Q(\mu_i) e^{-\beta
E(\mu_i)}}{\sum_{j=1}^M \textfrac{1}{p(\mu_j)} e^{-\beta
E(\mu_j)}}.
\end{equation}
Estos estimados convergen a los valores verdaderos cuando el número de datos $M$
converge al infinito.
 
\section{Muestreo de Boltzmann}
>Cuál distribución $p(\mu)$ nos dará los mejores resultados? Una posibilidad es la de intentar reproducir el sistema físico de la manera más fiel posible: muestreamos según la distribución de Boltzmann, $p(\mu) = \textfrac{1}{Z} e^{-\beta E(\mu)}$. Al sustituir esta distribución en el estimado, encontramos que
\begin{equation}
 \mean{Q} \simeq \frac{1}{M} \sum_{i=1}^M Q(\mu_i).
\end{equation}
Es decir, el promedio, tomando en cuenta las frecuencias de visita, se reduce a un promedio simple de los datos.

Sin embargo, esto nos lleva a otro problema: <no sabemos cómo generar distintas configuraciones según la distribución de Boltzmann!
La solución, sugerida por primera vez por Metropolis et al.~en los años 1950s, es que podemos generar esta --o cualquier otra-- distribución usando un \defn{proceso estocástico} que \emph{converja} hacia esta distribución. En particular, emplearemos normalmente una \defn{cadena de Markov}.

\section{Cadenas de Markov}
La idea es que generaremos una secuencia de configuraciones $\ss_t$ en el tiempo $t$. En cada paso, partiendo de la configuración $\ss$ actual, el sistema podrá escoger distintas configuraciones $\tt$ en el siguiente paso de tiempo $t+1$. Esta elección no será determinista, sino aleatorio: distintas configuraciones $\tt$ se podrán elegir con distintas \defn{probabilidades de transición} $P(\ss \to \tt)$.  Eso se repite en cada paso de tiempo (discreto), así dando lugar a un proceso estocástico. Para que sea una cadena de Markov, se requiere además que las probabilidades $P(\ss \to \tt)$ dependen solamente del estado actual $\ss$. Una caminata aleatoria normal es un ejemplo.

Denotemos por $w_\ss(t)$ la probabilidad de que el sistema se encuentre en el estado $\ss$ al tiempo $t$. Entonces tenemos que
\begin{equation}
 w_\tt(t+1) = \sum_{\ss} w_\ss(t) P(\ss \to \tt).
\end{equation}
Eso es, la probabilidad de estar en cierto estado en el siguiente paso de tiempo es igual a la suma de las probabilidades de que estaba en otra configuración en el paso anterior, por la probabilidad de que la nueva configuración se escogió.
Aunque la suma es, en principio, sobre todas las configuraciones del sistema, en la práctica se reduce a una suma sobre solamente las configuraciones de donde se puede alcanzar la configuración nueva $\tt$. 

Al examinar la forma del miembro de derecho en esta ecuación, nos percatamos de que tiene una forma parecida a una multiplicación de matrices. Por lo tanto, se puede reescribir como
\begin{equation}
 \ww(t+1) = \PP \cdot \ww(t).
\end{equation}
Nótese que tenemos que definir las componentes de la matriz $\PP$ como $P_{\tt \ss} \defeq P(\ss \to \tt)$ para que salga esta ecuación.

Al iterar esta relación, tenemos que
\begin{equation}
 \ww(t) = \PP^t \cdot \ww(0).
\end{equation}
Por lo tanto, el comportamiento asintótico en el tiempo de las probabilidades $w_\mu(t)$ depende de los valores y vectores propios de la matriz $\PP$ de probabilidades de transición.  Por ejemplo, si el valor propio más grande fuera menor que $1$ en módulo, entonces todas las probabilidades se decaerían a $0$. Sin embargo, dado que $\sum_\mu w_\mu(t) = 1$ para todo $t$, esto no es posible.
De hecho, esta condición de normalización implica que el valor propio de mayor módulo es $1$, y los demás valores propios son menores que $1$ en módulo.  Por lo tanto, la distribución converge a un límite $\ww(\infty)$ cuando $t \to \infty$, que satisface una invarianza bajo la acción de $\PP$:
\begin{equation}
w_\tt(\infty) = \sum_\ss w_\ss(\infty) P(\ss \to \tt)
\label{eq:eqm-distn-w}
\end{equation}
Es decir, esta distribución ya no cambia en el tiempo. Por lo tanto, la podemos considerar como una distribución de \defn{equilibrio}.
De hecho, es justamente la distribución --única-- de equilibrio $p(\ss)$ que genera esta cadena de Markov.

Distintas cadenas de Markov difieren en sus probabilidades de transición $P(\ss \to \tt)$. Al cambiar estas probabilidades de transición, se cambia en general la distribución de equilibrio que se genera. Para poder generar una distribución de equilibrio dada, es crucial, por lo tanto, averiguar cómo se relacionan la $\PP$ y la $p(\ss)$ que generar.


\section{Condiciones de balance detallado y ergodicidad}

Reescribiendo la ecuación \eqref{eq:eqm-distn-w} en términos de la distribución de equilibrio $p(\ss)$, obtenemos
\begin{equation}
 p(\tau) = \sum_\ss p(\ss) P(\ss \to \tau).
\end{equation}
Pero también podemos escribir el miembro de izquierda como
\begin{equation}
  p(\tau) = p(\tau) \sum_\ss P(\tau \to \ss) = \sum_\ss p(\tau) P(\tau \to \ss),
\end{equation}
donde la primera igualdad utiliza la normalización que la suma vale $1$ (``hay que ir a algún lado'').
Finalmente, llegamos a la \defn{condición de balance} 
\begin{equation}
 \sum_\ss p(\tau) P(\tau \to \ss) = \sum_\ss p(\ss) P(\ss \to \tau).
\end{equation}

La condición de balance es una condición \emph{necesaria} para que la distribución $p(\ss)$ sea una distribución invariante.
Una manera posible de satisfacer esta ecuación es igualando de manera individual los términos de los dos lados. De ahí resulta una condición
\emph{suficiente}, pero no necesaria, llamada la \defn{condición de balance detallado}:
\begin{equation}
p(\ss) P(\ss \to \tau) =  p(\tau) P(\tau \to \ss),
\end{equation}
que se tiene para \emph{todas} las configuraciones $\ss$ y $\tt$.
Podemos pensar que esta ecuación especifica que los flujos de probabilidad en equilibrio de una configuación a otra es igual al flujo de regreso.

En el caso del muestreo de Boltzmann, la distribución de equilibrio que nos interesa es justamente la de Boltzmann:

\subsubsection{Ergodicidad}
Otra condición importante que tiene que satisfacer el proceso es la de \defn{ergodicidad}. Eso requiere que cualquier configuración está accesible
desde cualquier otra, a través un número finito de pasos. Si no fuera el caso, entonces podría haber unas partes del espacio de configuraciones que nunca se
pudiera alcanzar.







