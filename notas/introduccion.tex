\chapter{La física estadística computacional}

La física estadística tiene como meta el establecer las propiedades a nivel
\emph{macroscópico} de un sistema compuesto por muchas partículas a nivel
\emph{microscópica}. Como se ven en los cursos básicos de esta materia, esta
meta se puede lograr en ciertos casos de manera exacta, como son los gases
ideales y los paramagnetos.

Pero rara vez se tratan en los cursos básicos sistemas que consisten en
partículas \emph{en interacción}. La razón por este hecho es sencillo: es, en
general, \emph{muy difícil} resolver problemas de este tipo. Sin embargo, esta
clase de sistemas es exactamente donde pasan los fenómenos de interés físico,
por ejemplo, las transiciones de fase, las propiedades emergentes, y las
propiedades de transporte.



\section{Métodos de Monte Carlo} 
Este curso se trata de otro enfoque para poder obtener información acerca de
sistemas de partículas en interacción --un enfoque \emph{computacional}.

Aún si las cantidades macroscópicas que nos interesan se pueden ver como
cantidades deterministas, del punto de vista de la física estadística, están
dadas por \emph{promedios} de distintos tipos sobre cantidades microscópicas
que fluctúan con cierta distribución de probabilidad.  

Por lo tanto, estudiaremos principalmente métodos tipo \emph{Monte Carlo} --así
llamados por emplear muchos números aleatorios para simular al
sistema\footnote{Monte Carlo es una ciudad en el principado de Monaco, en el sur
de Francia, famosa por el gran número de casinos que se encuentran ahí.}.


\section{Elección de idioma de programación}
Los métodos tipo Monte Carlo suelen requerir la repetición de ciertas
operaciones bastante sencillas millones de veces. Por lo tanto, es necesario
contar con un lenguaje de programación sumamente eficiente para llevar a cabo
simulaciones de este tipo.

Los dos candidatos más apropiados son Fortran 90 y C++. En este curso
emplearemos C++, que cuenta con ciertos métodos más modernos de programación.
Sin embargo, Fortran 90 es una buena opción también para este tipo de
aplicaciones. 

Unas razones por las cuales utilizamos C++ son los siguientes: es poderoso,
flexible, rápido, y capaz de imponer estructura en proyectos grandes. Se pueden
utilizar distintos estilos modernos de programación que permiten
  flexibilidad y eficiencia al mismo tiempo. Además, se está ocupando cada vez
más para los proyectos de cómputo científico en todo el mundo.

Algunas desvantajas de C++ son: su sintaxis puede ser algo complicada;
hay que recompilar después de cualquier cambio del
programa; hay mucho que aprender (porque contiene muchas posibilidades);
y los mensajes
de error pueden ser difíciles de entender.

\section{Entorno de análisis de datos}

Una vez que se hayan generado datos, es necesario analizar los datos generados.
Una buena opción moderna para eso es el idioma de programación interpretado
Python, en conjunto con distintos paquetes y librerías de software libre que
están disponibles para interactuar con Python.  También ocuparemos el programa
\texttt{gnuplot} para gráficas.

Todo el trabajo se llevará a cabo en un sistema operativo tipo Unix, que podría
ser Linux ó BSD (la base de MacOS, por ejemplo). Los entornos tipo Unix proveen
muchas herramientas de gran utilidad para el cómputo científico que no están
disponibles en otros sistemas operativos.  Todas las herramientas utilizadas en
el curso serán de \emph{software libre}.


\section{Metas del curso}
Las metas del curso son principalmente las siguientes dos:
\begin{itemize}
 \item Aprender a utilizar un idioma de programación potente y moderno; y
\item Desarrollar técnicas computacionales para estudiar problemas que surgen
en la física estadística.
\end{itemize}
