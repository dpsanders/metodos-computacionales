\chapter{Herencia y polimorfismo}

En este capítulo, llegaremos a discutir algunos de los conceptos y métodos centrales de la programación orientada a objetos: la herencia y el polimorfismo.

\section{Punteros}

Hasta ahora, hemos podido evitar utilizar (al menos de manera explícita) los punteros\footnote{A diferencia de C puro, donde los punteros juegan un papel básico para utilizar arreglos de tamaño variable, por ejemplo.}. Pero para los fines de este capítulo, sí es necesario emplearlos.

Los punteros funcionan de manera parecida a las referencias. Son objetos que ``apuntan a'' un objeto en la memoria: un puntero es una variable que contiene literalmente la dirección del espacio reservado en la memoria donde se guarda el contenido de otra variable.
Así que se puede examinar y modificar el valor de la variable de manera indirecta a través del puntero.

Un puntero solamente puede apuntar a un objeto de un solo tipo, así que hay punteros a \inl{int}, a \inl{double}, etc.  Tambien se pueden declarar punteros que
apuntan a un objeto de un tipo definido por el usuario, por ejemplo un puntero a objetos de tipo \inl{Particula}.

Un puntero a \inl{int} se declara con la siguiente sintaxis:
\begin{lstlisting}
int a = 3;
int* p;      // p se declara como un puntero a int
p = &a;   // el valor de p es la direccion de a

cout << "a = " << a << endl;
*p = 17;   // cambiar a 17 el contenido del objeto a lo cual apunta p
cout << "a = " << a << endl;
cout << *p << endl;   // 
\end{lstlisting}
Nótese que el asterisco tiene dos significados: indica que se está declarando un puntero si se utiliza en una declaración, mientras en cualquier otro contexto quiere decir ``el valor del objeto al cual apunta el puntero''. El ampersand (\inl{&}) se utiliza para asignarle la dirección de una variable a un puntero, para que el puntero apunte a dicha variable.

Si existe una clase \inl{Vec} (es decir, ya se ha declarado), entonces se puede declarar un puntero a \inl{Vec} de igual manera con
\begin{lstlisting}
Vec v;
Vec* pp;
pp = &v;
\end{lstlisting}
En este caso, si \inl{Vec} contiene una variable llamada \inl{x}, entonces se puede accesar con
\begin{lstlisting}
(*pp).x = 3.0;
\end{lstlisting}
Dado que esta sintaxis es complicada y fea, existe una alternativa equivalente más sugestiva:
\begin{lstlisting}
pp -> x = 3.0;
\end{lstlisting}
La flecha tiene el significado de ``modificar la variable \inl{x} en el objeto a lo cual apunto \inl{pp}''.

\section{Herencia}

Hasta ahora, hemos utilizado los objetos (clases) como un concepto que modela una colección de información y métodos. 
Esta idea de guardar distintos datos juntos adentro de un objeto se llama \defn{encapsulación}, y es el primer elemento clave de la programación orientada a objetos.

Otro elemento clave de la programación orientada a objetos es la \defn{herencia}. La herencia es un mecanismo que nos permite relacionar distintos tipos de objeto en jerarquías que expresan el concepto \defn{tipo de} --eso es, la herencia es una técnica que nos permite declarar que un objeto es un \emph{tipo de} otro objeto. 

Por ejemplo, >qué quiere decir que una manzana sea un \emph{tipo de} fruta, o que una laptop sea un \emph{tipo de} computadora? Quiere decir
que todas las frutas tienen ciertas propiedades \emph{en común}, y que en particular una manzana comparte estas propiedades, además de otras posibles propiedades que tenga. De igual manera, en C++ la herencia nos permite decir que un objeto A es un tipo de otro objeto B, y por lo tanto el objeto A \defn{hereda} todas las propiedades del objeto B, que son las propiedades comunes a todos los \defn{subtipos} de B, como es el caso de A.

\section{Ejemplo de herencia}
Supongamos que queremos escribir un programa que dibuje distintos tipos de formas en la pantalla, por ejemplo puntos, rectángulos, discos, y triángulos.
Cada una de estas formas tiene cierta información que es común a todas las formas, por ejemplo cada una tiene una posición y velocidad en dos dimensiones, un color, tal vez una masa, etc. Pero luego difieren en otra información, por ejemplo el radio del disco y los dos lados del rectángulo.
Por lo tanto, podemos pensar que hay un tipo de objeto básico, llamado \inl{Forma}, que tiene las propiedades comunes, y cada forma es un \emph{subtipo} de \inl{Forma}. Eso se puede implementar como sigue. La clase base se llama \inl{Forma}:
\begin{lstlisting}
class Forma {
  public:
	Vector posicion;
	Vector velocidad;
	int color;
	double masa;
}
\end{lstlisting}
Las variables de la clase base son públicas para que sean accesibles desde las otras clases que las heredan.
Ahora declaramos las otras formas para que \defn{hereden} la información común de \inl{Forma}:
\begin{lstlisting}
class Punto : public Forma {
};

class Rectangulo : public Forma {   
  private:
  double altura, anchura;
};

class Disco : public Forma {
  private:
	radio;
};
\end{lstlisting}
Nótese que \inl{Punto} no requiere ninguna información extra.

Ahora queremos que además cada objeto se puede mover, por lo cual agregamos un método (una función) a la clase \inl{Forma}.
Eso se puede hacer adentro de la declaración de la clase, o como sigue:
\begin{lstlisting}
Forma::mover(double dt) {
  posicion += dt * velocidad;	
  cout << "mover() de Forma" << endl;  
}
\end{lstlisting}

Ahora podemos mover a objetos de diferentes tipos:
\begin{lstlisting}
Punto p;
Disco d;

p.mover(0.1);
d.mover(0.1);
\end{lstlisting}

\section{Polimorfismo}
Ahora llegamos a la parte clave para la cual sirve la herencia: el \defn{polimorfismo}.
Queremos que un solo objeto pueda ser de diferentes tipos. Por ejemplo, queremos poder escoger si crear un punto, un rectángulo o un disco,
y luego moverlo --es decir, queremos un solo objeto que puede ser de cualquiera de estos tipos. Normalmente eso no se puede hacer en C++, ya que cada objeto puede ser de un solo tipo.

La solución es la de utilizar un puntero, que apuntu al tipo \emph{base} \inl{Forma}. Como los tipos \inl{Punto}, \inl{Rectangulo} etc.\ son subtipos de \inl{Forma}, un puntero a \inl{Forma} también puede apuntar a cualquiera de los subtipos, es decir los tipos \defn{derivados} de la clase base.
Funciona como sigue:
\begin{lstlisting}
Forma* p;    // puntero a Forma
Disco disco;
Rectangulo rect;

p = &disco;
p->mover();

p = &rect;
p->mover();
\end{lstlisting}
Vemos que \inl{p} apunta en turno a dos objetos de dos tipos diferentes, pero subtipos de \inl{Forma}, y que podemos entonces llamar a la función 
\inl{mover()} de cada objeto a través de un solo puntero \inl{p}. De cierta forma entonces, \inl{p} ``puede ser de diferentes tipos''.

\section{Los operadores \inl{new} y \inl{delete}}
Para evitar tener que crear objetos de cada tipo antes de utilizarlos, podemos hacer mano del operador \inl{new}. Este operador crea un objeto de cierto tipo, y regresa la dirección de la memoria reservada para contener este objeto, la cual se tiene que asignar a un puntero:
\begin{lstlisting}
Forma* p = new Disco;
Forma* p2 = new Rectangulo;

p->mover();
p2->mover();
\end{lstlisting}
Cuando terminemos de utilizar el objeto correspondiente, es necesario liberar la memoria que ocupa, con el operador \inl{delete}:
\begin{lstlisting}
delete p;
delete p2;
\end{lstlisting}

\section{Polimorfismo logrado: funciones virtuales}
Ahora supongamos que el rectángulo y el disco se mueven de \emph{distintas} maneras --por ejemplo, el rectángulo puede rotarse, mientras que para el disco no.
Entonces queremos definir distintas versiones de \inl{mover()} para \inl{Rectangulo} y \inl{Disco}:
\begin{lstlisting}
Rectangulo::mover() {
	cout << "Mover un rectangulo" << endl;
}

Disco::mover() {
	cout << "Mover un disco" << endl;
}
\end{lstlisting}
Pero si intentamos mover los dos objetos usando un puntero, nos da una sorpresa
\begin{lstlisting}
Forma *p = new Rectangulo;
p -> mover();
delete p;

Forma *p = new Disco;
p -> mover();
delete p;
\end{lstlisting}
Encontramos que la función que se llama es la \inl{mover()} original, aquella definida en \inl{Forma}.
Sin embargo, no nos debería de sorprender tanto, ya que \inl{p} es justamente un puntero a \inl{Forma}.

Para lograr el comportamiento deseado, donde se llama a la versión adecuada de la función \inl{mover()} para cada subtipo, es necesario agregar una sola palabra: \inl{virtual}. Es decir, la función \inl{mover()} \emph{en la clase base} se declara como una función ``virtual'', como sigue:
\begin{lstlisting}
virtual Forma::mover() {
	cout << "mover() de Forma" << endl;
}
\end{lstlisting}

Ahora si volvemos a llamar a \inl{mover()} de los subtipos, sí llama a la versión definida para cada uno de ellos.
Es decir, \emph{en el momento de correr el programa}, se decide cuál versión de \inl{mover()} se tiene que llamar, dependiendo del tipo de \inl{p} \emph{en este momento}. 

\ejercicio
Escribe un programa sencillo de dibujo.
Segun lo que teclee el usuario --'r' o 'd'-- crea cualquier número de objetos de tipo \inl{Rectangulo} y \inl{Disco}, hasta que el usuario teclee
't' (terminado). Dibuja los objetos con \texttt{gnuplot}.  Las posiciones de los objetos pueden ser aleatorios.

\section{Clases base abstractas}
Aún cuando una función en una clase base es una función \inl{virtual}, no es necesario proveer ninguna definición nueva de la función en una subclase. Si no se provee para un subtipo dado, entonces este subtipo utilizará la versión definida en la clase base.

Sin embargo, a veces es útil \emph{obligar} a cada subclase proveer su propia función. 
Esto se puede hacer al declarar la función virtual en la clase base como sigue:
\begin{lstlisting}
class FormaAbstracta {
  public:
	virtual void mover() = 0;
};
\end{lstlisting}
El \inl{= 0} es una manera de informarle al compilador que la función \inl{mover()} \emph{no está definida} en la clase base.
Por lo tanto, \emph{no es posible crear ningún objeto de tipo \inl{Forma}}. Además, cualquier subtipo debe proporcionar su propia definición de \inl{mover()}. Solamente se pueden definir objetos de estos subtipos.
Por lo tanto, la clase base es una clase base \defn{abstracta}.


