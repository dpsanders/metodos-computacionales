\chapter{El modelo de Ising y la física estadística}
En este capítulo, introduciremos un modelo básico en la física estadística, el
\defn{modelo de Ising}, y reseñaremos los procedimientos básicos de la física
estadística.  

La meta de la física estadística es la de calcular las propiedades
macroscópicas y termodinámicas de un sistema, a partir de una descripción
microscópica del mismo.  Por lo tanto, para cada modelo, cabe especificarlo al
dar su \defn{Hamiltoniano}, es decir, la función que determina la energía total
del sistema cuando éste se encuentra en un microestado dado.

\section{El modelo de Ising}
El modelo de Ising es interesante ya que es un modelo sumamente
sencillo de plantear, pero que exhibe distintos fenómenos de interés, tales
como las transiciones de fase.  Es un modelo de un imán cristalino, y consiste
en muchas partículas, llamadas \defn{espines}, que se encuentran en lugares
fijos en una red cristalina. Los espines pueden tomar, en el caso más sencillo,
dos valores diferentes, apuntándose hacia arriba o hacia abajo.  

Los espines
modelan los momentos magnéticos con los electrones sin pareja en un metal
ferromagnético como el hierro. En tales materiales, hay una interacción entre
espines cercanos, llamada la \defn{interacción de intercambio}, que implica que
éstos tienen una tendencia a alinearse con sus vecinos, ya que eso es un estado
más energéticamente favorable.

Para convertir esta descripción en un modelo, consideremos una red de $N$
sitios, con etiquetas $i=1, \ldots, N$. En cada sitio $i$ de la red hay un
espín $\s_i \in \{\pm1\}$, es decir, cada espín puede tomar los valores $+1$ y
$-1$ (arriba y abajo, respectivamente).  La configuración completa del sistema
se denota por $\ss \defeq (\s_i)_{i=1,\ldots,N}$.

Para especificar la energía, consideraremos solamente interacciones entre pares
de espines, especificadas a través de la energía de interacción $E(i,j)$ entre
los espines $\s_i$ y $\s_j$.  Si los espines tienen el mismo valor, entonces
asignaremos una energía favorable $E(i,j) = -J$, mientras que si tienen valores
opuestos, la energía es $E(i,j) = J$. Además, solamente los espines
\emph{vecinos} podrán interactuar. 

Entonces tenemos
\begin{equation}
E(i,j) = \begin{cases} -J \s_i \s_j, & \text{si $i$ y $j$ son sitios vecinos;}\\
          0, & \text{si no.}
         \end{cases}
\end{equation}
Nótese que el producto $\s_i \s_j$ da justamente $1$ si los espines están
alineados, y $-1$ si no.

Por lo tanto la energía total de una configuración es
\begin{equation}
\HH(\ss) = E(\ss) \defeq -J \sum_{\langle i, j \rangle} \s_i \s_j.
\end{equation}
Aquí, la notación $\langle \cdot \rangle$ denota una suma sobre pares de sitios
vecinos en la red. Hacemos una distinción entre la energía de interacción
$E(\ss)$, y la energía total del sistema $\HH(\ss)$. En este caso, no hay otra
contribución a la energía, así que el Hamiltoniano está dado por la energía de
interacción entre espines.
Podemos pensar en esta suma como una suma sobre los
\emph{enlaces} entre los sitios.

\section{Física estadística}
Ya que hemos especificado un modelo de manera microscópica, la física
estadística da una ``receta'' para calcular sus propiedades termodinámicas
macroscópicas. Boltzmann y Gibbs mostraron que si ponemos nuestro sistema
microscópico en contacto con un baño térmico a temperatura $T$, pero tal que no
puede intercambiar materia con el baño, entonces el sistema se puede modelar
utilizando el \defn{ensamble}\footnote{También llamado ``conjunto
representativo''.} \defn{canónico}, en donde
la frecuencia con la cual el
sistema visita una configuración $\ss$ dada es
\begin{equation}
 p(\ss) \propto e^{-\beta \HH(\ss)},
\end{equation}
donde $\beta \defeq 1/(k_{B} T)$ es la temperatura inversa, y $k_B$ es la
constante de Boltzmann. Tomaremos siempre unidades para las cuales $k_B = 1$.

Normalizando la distribución de probabilidad, usando que $\sum_{\ss \in \Omega}
p(\ss) = 1$, donde $\Omega$ es el conjunto de estados posibles del sistema,
obtenemos que
\begin{equation}
 p(\ss) = \frac{1}{Z(\beta)} e^{-\beta \HH(\ss)},
\end{equation}
donde 
\begin{equation}
Z(\beta) \defeq \sum_{\ss \in \Omega} e^{-\beta \HH(\ss)}
\end{equation}
se llama la \defn{función de partición}, ya que involucra la manera en la
cual se distribuye, o particiona, la probabilidad entre los distintos
microestados del sistema.  El promedio de un observable (es decir, una
cantidad que podemos medir en el sistema) $Q$ está dado por
\begin{equation}
 \mean{Q} \defeq \sum_{\ss \in \Omega} Q(\ss) p(\ss).
\end{equation}


Si conocemos $Z(\beta)$, entonces podemos derivar todas las cantidades
termodinámicas. Por ejemplo, la energía interna macroscópica $U$ se identifica
--en el \defn{límite termodinámico} $N\to \infty$, $V \to \infty$ con $\rho
\defeq N/V$ fija-- con el promedio microscópico $\mean{E}$. Pero
\begin{equation}
\mean{E} =  \sum_{\ss} E(\ss) p(\ss) = \frac{1}{Z} E(\ss) e^{-\beta E(\ss)}  =
-\frac{1}{Z} \dd{Z}{\beta} = -\dd{}{\beta} \log Z(\beta).
\end{equation}
Además, podemos decir que la energía libre $F(\beta)$ está dada por
\begin{equation}
 F(\beta) = -k_B T \log Z(\beta).
\end{equation}
De una manera similar, las cantidades macroscópicas termodinámicas se pueden
escribir como funciones y derivadas de la función de partición $Z(\beta)$.

\section{Los cálculos sin imposibles}
En principio, la receta que provee la física estadística nos permite calcular
cualquier cantidad macroscópica deseada, a partir del conocimiento del
Hamiltoniano y la función de partición de un sistema. Sin embargo, en la
práctica esta esperanza no se cumple, ya que los cálculos requeridos son
\emph{intratables}, debido al siguiente problema combinatórico.

Consideremos el modelo de Ising con $N$ espines. El número total de
configuraciones del sistema es $|\Omega| = 2^N$. Para un sistema pequeño, de
$100 \times 100$ espines en dos dimensiones, tenemos que $N=10000$, y por lo
tanto $|\Omega| = 2^{10000} \simeq 10^{3000}$!  Este número es realmente
enorme, <aproximadamente $10^{3000}$ veces la edad del universo en segundos!
Por lo tanto, no importa el tamaño de la computadora que tengamos, nunca
podremos llevar al cabo el cálculo de la función de partición ``exacta''.

La solución a esto es la de aproximar las sumas con sumas sobre un número
reducido de configuraciones. Sin embargo, eso funcionará solamente si las
configuraciones sobre las cuales se suma se escogen de manera apropiada para el
problema, para que sean las ``relevantes'' o  ``dominantes''.  La sigiuente
parte del curso se tratará de las maneras de resolver este problema.





