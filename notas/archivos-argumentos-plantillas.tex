\chapter{Referencias, archivos, argumentos de la línea de comandos y plantillas
de función}

En este capítulo, veremos algunas técnicas de C++ básico que nos ayudan a
producir datos bien estructurados.

\section{Calcular promedios en una función}
Supongamos que nuestro programa produce unos datos, y quisiéramos calcular el
promedio y otras propiedades estadísticas de estos datos. Tenemos tres
posibilidades:
\begin{enumerate}
\item Actualizar una suma corrida cada vez que generamos un dato, y calcular el
promedio al final;
\item Guardar los datos en un arreglo mientras el programa corre, y calcular
las propiedades al final;
\item Guardar los datos en un archivo en el disco, y procesarlos posteriormente.
\end{enumerate}

La primera opción tiene la ventaja que está más sencillo, pero se puede volver
complicado calcular muchas cantidades, y hay que modificar el código adentro
del programa para hacerlo. La tercera opción es tal vez la mejor, ya que los
datos ``crudos'' están disponibles para calcular nuevas cantidades
posteriormente.

Por el momento, consideremos la opción dos. Mientras generamos los datos, los
vamos guardando en un arreglo, que en C++ se podría implementar con un
\inl{vector}:
\begin{lstlisting}
vector<int> pos;
pos.push_back(3);
\end{lstlisting}
Podemos utilizar el método \inl{push_back()} del \inl{vector} para agregar más
datos en cualquier momento.

\ejercicio
Los vectores tienen un método \inl{capacity()} que reporta la cantidad de
espacio que está reservado actualmente para que crezca el vector. Investiga
cómo cambia esta capacidad al agregar cada vez más elementos con
\inl{push_back()}.

Al final del programa querramos calcular, por ejemplo, un promedio.
Para poder reutilizar este código, quisiéramos ponerlo en una función:
\begin{lstlisting}
double promedio(vector<int> v) {
  int suma = 0;
  for (int i=0; i < v.size(); i++) {
    suma += v[i];
  }
  return suma / double(v.size());
}
\end{lstlisting}
Nótese que pasamos un vector completo como argumento a la función.

\section{Referencias}
Para entender mejor el efecto de pasar argumentos a funciones, consideremos un
caso muy sencillo:
\begin{lstlisting}
void f(int a) {
  cout << "En f(): a = " << a << endl;
  a = 3;
  cout << "En f(): a = " << a << endl;
}

int main() {
  int b = 7;
  cout << "En main(): b = " << b << endl;
  f(b);
  cout << "En main(): b = " << b << endl;
}
\end{lstlisting}
>Cuál será la salida de este programa? Podríamos esperar que el valor de
\inl{b} cambiaría, ya que es el argumento de la función \inl{f()}, donde se
modifica. Sin embargo, eso no es cierto --el valor de \inl{b} no cambia.

La razón por esto es que la variable \inl{a} se crea como una variable nueva
que existe nada más adentro de la función \inl{f()} --es decir, es una variable
\defn{local}. Su valor es una \emph{copia} del valor de \inl{b}, y por lo 
tanto, cuando se le asigna un nuevo valor, no afecta el valor de \inl{b}.

Sin embargo, hay situaciones en las que \emph{sí} queremos que las funciones
modifiquen a las variables externas de ellas. Eso se logra utilizando una
\defn{referencia}:
\begin{lstlisting}
void f(int& a) {
  a = 3;
}
\end{lstlisting}
Una referencia, declarada usando un ámpersand \inl{&} después del tipo de
la variable, declara un ``alias'', u otro nombre, de una misma variable. En
este caso, \inl{a} se vuelve otro nombre para \inl{b} --se puede considerar
como un \defn{puntero} a \inl{b}-- y, por lo tanto, cuando se modifica \inl{a}
es ahora equivalente de modificar \inl{b}.

Regresemos al caso de la función \inl{promedio} de la sección anterior.  Lo que
acabamos de ver es que cuando declaramos la función usando
\begin{lstlisting}
double promedio(vector<int> v) {
}
\end{lstlisting}
el vector \inl{v} será una \emph{copia} del vector original. Se copiarán todos
los datos a un nuevo \inl{vector}, solamente para calcular su promedio. Esto se
puede evitar al utilizar una referencia, así que la declaración se vuelve
\begin{lstlisting}
double promedio(vector<int>& v) {
}
\end{lstlisting}
Ahora \inl{v} es simplemente otro nombre para el mismo objeto, y ya no se copia
ninguna información.

Sin embargo, esto permitiría modificar a los datos originales, que tampoco
queremos. Para evitar eso, ponemos la declaración siguiente:
\begin{lstlisting}
double promedio(const vector<int>& v) {
}
\end{lstlisting}
La palabra \inl{const} (``constante'') indica al compilador que esta función
\emph{no} está permitida modificar el contenido del objeto.





